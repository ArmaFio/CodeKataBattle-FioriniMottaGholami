\documentclass{article}
\usepackage{graphicx} % Required for inserting images
\usepackage{titlesec}
\usepackage{tabularx}
\usepackage[none]{hyphenat}
\setcounter{secnumdepth}{4}

\titleformat{\paragraph}
{\normalfont\normalsize\bfseries}{\theparagraph}{1em}{}
\titlespacing*{\paragraph}
{0pt}{3.25ex plus 1ex minus .2ex}{1.5ex plus .2ex}


\title{\textbf{REQUIREMENT ANALYSIS AND SPECIFICATION DOCUMENT}\\\textit{CODE KATA BATTLE}}
\author{Armando Fiorini, Samuele Motta, Vajihe Gholami}
\date{}
\begin{document}
\maketitle
\section*{INFO}
\textbf{Deliverable}: RASD\\
\textbf{Title}: Requirement Analysis and Verification Document\\
\textbf{Authors:} Armando Fiorini, Samuele Motta, Vajihe Gholami\\
\textbf{Version:} 0.0.0\\
\textbf{Date}: 18/10/2023\\
\textbf{Download page}: https://github.com/ArmaFio/FioriniMottaGholami\\
\newpage
\tableofcontents
\newpage


\section{Introduction}
\subsection{Purpose}
The purpose of this document is to outline the functionalities and requirements of CodeKataBattle (CKB), a novel platform designed to facilitate the enhancement of students' software development skills through collaborative training on code katas. Educators leverage this platform to challenge and mentor students by creating code kata battles, where teams of students engage in friendly competitions to showcase and enhance their programming skills.


\subsection{Scope}
Nowadays the world of computer science is more than ever important and present in everyone's life. For this reason, it's crucial for the teaching system to have the best
possible means to lead the students to the best possible comprehension and governance of this subject. \\
CodeKataBattle is a platform supposed to allow Educators to create coding tournaments in which students can compete, in teams or on their own, to improve their skills in any type of programming language.\\\\
In order to ensure that the system allows Educators to manage all the settings and rules of the tournaments (number of battles, allowed Students' team size, score assignment rules, proper programming language \dots), to which students can subscribe.\\
All the Students logged into the app receive a notification each time a tournament has been created.\\
Students involved in a tournament can create teams (according to the tournament's rules), and then it's time for them to start coding: through Git they will be allowed to receive the problem's text and to submit their solution, within a configured time range.\\
The solutions will then be evaluated, by the system or the Educator, and will contribute to defining the final rank of the tournament.\\\\
Their stats in tournaments will be public and visible in the account of a student, as well as their achieved badges: a badge is a reward that is created by an Educator, who defines the requirement(s) to gain it: the badges can be assigned to one or more students and are relative to a single tournament, at the end of which they are eventually assigned.

\subsection{Definitions, Acronyms, Abbreviations\\} 
\subsubsection{Definitions\\}
\textbf{Educator:} user who signs up to use the system as a mean to improve the programming skills of his students, he can create Tournaments
and badges.\\\\
\textbf{\\Student:} user who signs up to improve their skills, participating to tournaments which are created by the educators. \\\\\\
\textbf{\\Tournament:} coding challenge, consisting in a certain number of battles.\\\\\\
\textbf{\\Team:} group of students, formed to join a tournament and work together to win the battles.\\\\\\
\textbf{\\Battle:} every single challenge the tournament is composed from. \\\\\\
\textbf{\\(Gamification) Badge:} achievement that is created from an educator, and can be obtained from Students satisfacting the estabilished requirements.\\\\
\textbf{}
\subsubsection{Acronyms}
\begin{itemize}
    \item \textbf{CKB}: CodeKataBattle
    \item \textbf{RASD}: Requirement Analysis and Specification Document
    \item \textbf{UI}: User Interface
    \item \textbf{UML}: Unified Modelling Language
\end{itemize}

\subsubsection{Abbreviations}

\subsection{Overview}
\subsubsection{Goals}
These are the goals that the system is supposed to achieve:\\\\
$[G1]$ Educators and students can subscribe to the application creating their personal account \\
$[G2]$ Educators create Tournaments in which students can compete \\
$[G3]$ Students subscribe to the tournaments \\
$[G4]$ Students compete, in teams or on their own, in many battles, the results of which will determine the final rank of the tournament  \\
$[G5]$ Students can reach achievements that permit them to gain badges for their account \\
$[G6]$ Educators create badges and define requirements to achieve them \\
\begin{itemize}
\item Skill Enhancement: Users, primarily students, want to improve their software development skills by actively participating in coding battles and honing their problem-solving abilities. 

\item Collaboration: Students aspire to collaborate with their peers to solve coding challenges as a team. This encourages cooperative learning and effective communication among team members.

\item Competition and Ranking: Users, including educators and students, desire a competitive element in software development exercises. They aim to compete with others and achieve high rankings based on their coding prowess.

\item Educational Engagement: Educators intend to create engaging and effective teaching environments, using code kata battles to educate and mentor their students in a practical and enjoyable manner.

\item Automated Assessment: All users seek an automated and objective method for assessing and scoring coding solutions. This provides fair and efficient evaluation, particularly in cases where a large number of students and teams are involved.

\end{itemize}
\newpage
      \subsubsection{Phenomena}
        {\setlength{\leftskip}{2em}
            \paragraph{World Phenomena}
            $[W1]$ User downloads the application\\
            $[W2]$ An Educator decides in class to use the application to organize a coding tournament\\
            $[W3]$ A Student decides to join a battle\\
            $[W4]$ A Student contacts some mates to form a team

            \paragraph{Shared Phenomena}
            \textbf{Controlled by the world and observed by the machine} \\\\
            $[SW1]$ A user creates an account and logs into the application\\
            $[SW2]$ An Educator creates a tournament\\
            $[SW3]$ A Student, or a team of students, joins a tournament\\
            $[SW4]$ An Educator creates a badge\\
            $[SW5]$ A Student or a team submits a solution to the given coding problem\\
            $[SW6]$ A User decides to visualize his or another one's account page
\\\\
            \textbf{Controlled by the machine and observed by the world}\\\\
            $[SM1]$ Student receives a notification: a tournament has been created \\
            $[SM2]$ A Student is notified that the time for submitting their solution is terminating\\
            $[SM3]$ A Student is shown he has achieved a badge\\
            $[SM4]$ A User is shown his account or another one's, with all the badges and stats

            }

\newpage

\section{Overall Description}

\subsection{Assumptions, dependencies and constraints}

\subsubsection{Domain Assumptions}
The following are the domain assumptions that need to be satisfied for the correct behaviour of CKB. Since those situations are out of the control of the system, they are taken as granted.\\\\
$[D1]$ User must have an internet connection.\\
$[D2]$ User must have a GitHub account.\\
$[D3]$ User must be able to push on GitHub by itself.\\
$[D4]$ Connection between systems must be reliable.\\

\section{Specific Requirements}

\subsection{User Interfaces}
In this section is presented the user interface through which both students and educators can:
\begin{itemize}
    \item Log in
    \item Sign Up
    \item Access personal dashboard
\end{itemize}
The needs of educators and students are not the same, therefore they are going to get access to different views:\\
Educators have access to specific views that allow them to create tournaments and battles, moreover, the creator of the tournament can grant access to other educators so that they can create their own battles within the tournament access has been granted for.\\
An user can be both student and educator: those users will be provided with a switch button available in the dashboard so that they can switch between the two profiles.
\begin{center}
    \textbf{ADD IMAGES HERE}\\
\end{center}

\subsection{Requirements}
\subsubsection{Functional Requirments}
$[FR1]$ The system allows Users (Students and Educators) to sign up
$[FR2]$ The system allows Users to (Students and Educators) to login
$[FR3]$ The system allows Educators to create tournaments
$[FR4]$ The system allows Educators to set the maximumn number of students per team in a tournament
$[FR5]$ The system allows Educators to set the number of battles in a tournament
$[FR6]$ The system allows Educators to create battles
$[FR7]$ The system allows Educators to choose whether and when the scores will be assigned automatically by the system or manually by themselves.
$[FR8]$ The system allows Educstors to create badges for a specific tournament
$[FR9]$ The system allows Educators to state the requirements for achieveing a badge
$[FR10]$ The system allows Users to join a tournament
$[FR11]$ The system allows Users to see the ranking in tournaments and achieved badges of every signed up student in the system.
$[FR12]$ The system allows Educators to see the list of the active tournaments they have created
$[FR13]$ The system allows Users to see the list of the active tournaments they have joined
$[FR14]$ 

\newpage

\section{Alloy}
\includegraphics[scale = 0.7]{Untitled-1.pdf}

\section{Effort Spent}
\begin{center}
\textbf{Armando Fiorini} \\
\vspace{10px}
    \begin{tabularx}{0.8\textwidth} { 
  | >{\centering\arraybackslash}X 
  | >{\centering\arraybackslash}X | }
 \hline
 \textbf{Chapter} & \textbf{Hours Spent} \\
 \hline
 1 & TBD  \\
 \hline
 2 & TBD \\
 \hline
\end{tabularx}

\vspace{10px}
\textbf{Samuele Motta} \\
\vspace{10px}
\begin{tabularx}{0.8\textwidth} { 
  | >{\centering\arraybackslash}X 
  | >{\centering\arraybackslash}X | }
 \hline
 \textbf{Chapter} & \textbf{Hours Spent} \\
 \hline
 1 & TBD  \\
 \hline
 2 & TBD \\
 \hline
\end{tabularx}

\vspace{10px}
\textbf{Vajihe Gholami} \\
\vspace{10px}
\begin{tabularx}{0.8\textwidth} { 
  | >{\centering\arraybackslash}X 
  | >{\centering\arraybackslash}X | }
 \hline
 \textbf{Chapter} & \textbf{Hours Spent} \\
 \hline
 1 & TBD  \\
 \hline
 2 & TBD \\
 \hline
\end{tabularx}

\end{center}

\section{References}
\end{document}
